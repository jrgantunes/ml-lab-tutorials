\documentclass[12pt, a4paper, twoside]{article}
\usepackage[T1]{fontenc}
\usepackage[utf8]{inputenc}
\usepackage{authblk}
\usepackage{amsmath}

% GD and FB will define the authors order bassed on the publish field rules
\title{Finance Reinforcement Learning}
\author[*]{Fernando Bação}
\author[*]{Georgios Douzas}
\author[*]{Jorge Antunes}
\affil[*]{Information Management School - NOVA IMS}
\date{June 2019}

\renewcommand\Authands{ and }

\begin{document}


\maketitle

\begin{abstract}
	This is a simple paragraph at the beginning of the document. Here we will define the main goals and achievements of our work.
\end{abstract}


\section{Introduction}

Here we will right about all topics concerning our paper

\section{Move on}

Here will be all the remaining things

The sections will be provided by GD


\section{Appendix}

This section will be used internal mainly\par

Here we will find a group of decisions that will help us to define the problem and the \textit{reproducibility} of the article
\textbf{An adaptive portfolio trading system: A risk-return portfolio optimization using recurrent reinforcement learning with expected maximum drawdown. Almahdi, Yang 2017 }\\


The Sharpe Ratio is our basis metric once it measures both profitability and risk exposure. It is defined as $ Shrp = \frac{\gamma}{\sigma}$\\

Where $\gamma$ is the mean of the returns and $\sigma$ is the standard deviation of the returns.\\
% How many n returns? a decision to be made

The trading Signal is $F_t \in \{-1,1\}$\\
The relevance of the trading signal is the discount of the transactions, for sake of simplicity we may ignore them bu in real world, typically, a fee is applied

The log return is defined as $ r_t = log(price_t) - log(price_{t-1})$\\

Return Calculation (at time $ t $)
$ R_t = \mu \times [F_{t-1}.r_t - \delta|F_t-F_{t-1}|]$\\
where $\mu$ is the number of shares\\
$\delta$ is the transaction cost\\

Goal maximize the Sharpe Ratio which can be rewritten as $ S_T = \dfrac{E[R_T]}{\sigma}$\\

Based on Moody et al 1998 and Almahdi, Yang 2017\\

$ A = \frac{1}{T}\sum_{t=1}^{T}R_t $\\

$ B = \frac{1}{T}\sum_{t=1}^{T}R_{t}^{2} $\\

As eq. 5 in Almahdi and Yang 2017, I will repeat it to the Calmar Ratio\\

$$ \frac{dC_t}{d\theta} = \sum_{t=1}^{T}  \{\frac{dC_T}{dA}\frac{dA}{dR_t}+\frac{dC_T}{dB}\frac{dB}{dR_t}\}.\{\frac{dR_t}{dF_t}\frac{dF_t}{d\theta}+\frac{dR_t}{dF_{t-1}}\frac{dF_{t-1}}{d\theta}\} $$\\

The derivatives here are the following:\\

$ \dfrac{dR_t}{dF_t} = -\mu\delta.sgn(F_t - F_{t-1}) $\\

$ \dfrac{dR_t}{dF_{t-1}} = -\mu.r_t+\mu\delta.sgn(F_t - F_{t-1}) $\\

$ \dfrac{dF_t}{d\theta} = (1-tanh(x_{t}^{'}\theta)^2).(x_t + \theta_{M+2}\frac{dF_{t-1}}{d\theta}) $\\

eqivalent to\\

$ \dfrac{dF_t}{d\theta} = (1-\bigg(\dfrac{\mathrm{e}^{2x_{t}^{'}\theta}-1}{\mathrm{e}^{2x_{t}^{'}\theta}+1}\bigg)^2).(x_t + \theta_{M+2}\frac{dF_{t-1}}{d\theta}) $\\

Knowing that $ C_T  = \dfrac{TShrp^2}{0.63519 + 0.5\log T + \log Shrp}$\\

and Sharpe Ratio\\

$ S_T = \dfrac{\text{Average}(R_t)}{\text{Standard deviation}(R_t)} $\\

Our Sharpe Ratio for $ n $ returns is (Moody 1998)\\

$ S_n = \dfrac{A_n}{K_n\sqrt{B_n - A_{n}^{2}}} $ with\\

$ A_n = \dfrac{1}{n}\sum_{i=1}^{n}R_i $	$ B_n = \dfrac{1}{n}\sum_{i=1}^{n}R_{i}^{2} $	$ K_n = \bigg(\dfrac{n}{n-1}\bigg)^{\frac{1}{2}} $\\

The Calmar Ratio will be (drop $K_n$ for sake of simplicity), we ca rewrite it as\\



$$ C_T  = \dfrac{T\bigg(\dfrac{A_n}{\sqrt{B_n - A_{n}^{2}}}\bigg)^2}{0.63519 + 0.5\log T + \log \bigg(\dfrac{A_n}{\sqrt{B_n - A_{n}^{2}}}\bigg)}$$\\

So, $\dfrac{dC_T}{dA_n} = \dfrac{d\Bigg(\dfrac{T\bigg(\dfrac{A_n}{\sqrt{B_n - A_{n}^{2}}}\bigg)^2}{0.63519 + 0.5\log T + \log \bigg(\dfrac{A_n}{\sqrt{B_n - A_{n}^{2}}}\bigg)}\Bigg)}{dA_n}$\\
\\

Simplifing the Calmar Ratio $C_T$\\

$\dfrac{T\bigg(\dfrac{A_n}{\sqrt{B_n - A_{n}^{2}}}\bigg)^2}{0.63519 + 0.5\log T + \log \bigg(\dfrac{A_n}{\sqrt{B_n - A_{n}^{2}}}\bigg)}$\\
$ \equiv$
$\dfrac{TA_{n}^{2}}{(B_n - A_{n}^{2})\bigg(\log \bigg(\dfrac{A_n}{\sqrt{B_n - A_{n}^{2}}}\bigg)+ 0.5\log T + 0.63519\bigg)}$\\

Through the chain rule application, we obtain:\\

$$ \dfrac{dC_T}{dA_n} = \dfrac{200000TB_nA_n\left(100000\ln\left(\frac{A_n}{\sqrt{B_n-A_{n}^{2}}}\right)+50000\ln\left(T\right)+13519\right)}{\left(A_{n}^{2}-B_n\right)^2\left(100000\ln\left(\frac{A_n}{\sqrt{B_n-A_{n}^{2}}}\right)+50000\ln\left(T\right)+63519\right)^2} $$\\

$$ \dfrac{dC_T}{dB_n} = -\dfrac{100000TA_{n}^{2}\left(100000\ln\left(\frac{A_n}{\sqrt{B_n-A_{n}^{2}}}\right)+50000\ln\left(T\right)+13519\right)}{\left(B_n-A_{n}^{2}\right)^2\left(100000\ln\left(\frac{A_n}{\sqrt{B_n-A_{n}^{2}}}\right)+50000\ln\left(T\right)+63519\right)^2}$$\\


$ \dfrac{dA_n}{dR_t} = \dfrac{1}{n}$\\

$ \dfrac{dB_n}{dR_t} = \dfrac{2R_t}{n}$\\



\end{document}