\documentclass[12pt, a4paper, twoside]{article}
\usepackage[T1]{fontenc}
\usepackage[utf8]{inputenc}
\usepackage{authblk}
\usepackage{amsmath}


\title{Finance Reinforcement Learning}
\author[*]{Fernando Bação}
\author[*]{Georgios Douzas}
\author[*]{Jorge Antunes}
\affil[*]{Information Management School - NOVA IMS}
\date{June 2019}

\renewcommand\Authands{ and }

\begin{document}


\maketitle

\begin{abstract}
	This is a simple paragraph at the beginning of the document. Here we will define the main goals and achievements of our work.
\end{abstract}


\section{Introduction}


\section{Go on}


In this document some extra packages and parameters
were added. There is an encoding package,
and pagesize and fontsize parameters.

This line will start a second paragraph. And I can
brake\\ the lines \\ and continue on a new line.\par

The limit of $ f(x) $ as $ x $ approaches  $ x_0 $ is \textbf{the number}  $ L $ if the following criterion holds:
Given any radius $ \epsilon > 0 $ about $ L $, there exist a radius $ \delta $ about $ x_0 $ such that for all $ x $,
\[ 0<|x-x_0|<\delta \mbox{ implies } |f(x)-L|<\epsilon \]
If a function $ f(x) $ has a limit $ L $ as $ x $ approaches $ x_0 $,
it is denoted by
\[ \lim\limits_{x\to x_0} f(x)=L \]

\section{Appendix}

The Sharpe Ratio\\

$ \dfrac{\mathrm{e}^x\cos\left(\frac{\sqrt{\mathrm{e}^x+a}}{2}\right)}{4\sqrt{\mathrm{e}^x+a}} $


\end{document}